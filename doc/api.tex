\chapter{API for the rescue system}

\section{Architecture of the rescue system}
The rescue system is composed of a central framework and the modules for
error-detection and repair. The framework consists of the files\\

\begin{tabular}{ll}
\verb+src/osr.ycp+ & \verb+src/osr_popup.ycp+\\
\verb+src/osr_provides.ycp+ & \verb+src/osr_module_progress.ycp+\\
\verb+src/osr_system.ycp+ & \verb+src/osr_global_progress.ycp+\\
\verb+src/osr_api.ycp+ & \verb+src/osr_log_file.ycp+\\
\verb+src/osr_summary.ycp+ & \verb+src/osr_module_list.ycp+\\\\
\end{tabular}

Module files are\\

\begin{tabular}{ll}
\verb+src/osr_module_init.ycp+ & \verb+src/osr_module_lilo.ycp+\\
\end{tabular}

\section{Template module}
All modules should obey the standard module \verb+osr_module_lilo+ to be consistent
with the whole rescue system.

\section{Developing a new module}
A new module has the name \verb+osr_module_<name>+ and is placed in 
\verb+y2c_osr/src/osr_module_<name>.ycp+ (the module \verb+osr_module_lilo+ is placed in
the file \verb+y2c_osr/src/osr_module_lilo.ycp+).\\

To make the new module present to the framework a new
entry like \verb+`osr_module_<name>()+ has to be added to the file
\verb+osr_module_list.ycp+. This file contains all detection modules and
is read by the framework on startup.

\section{Include}
If the module needs special functions from outside the rescue-system it
can include other YCP-files. Some global YCP-files are already included
by the framework and so they are present to all modules. Other that than
mentioned above these included files are:\\

\begin{tabular}{ll}
\verb+ui/common_popups.ycp+ & \verb+ui/wizard_dialog.ycp+\\
\end{tabular}

\section{Control interface \& module-map}
For determining the sequence of the detection-methods and getting some
additional information the framework executes the initial functions of
each module where the module-map is created and returned to the framework.
For the LILO-module this is the function \verb+OSRLiloInit()+.\\

\subsection{Module name and display information}

{\footnotesize \begin{verbatim}
map osr_module_lilo = $[
     // has to be the name of the file
     "name"                  :  "osr_module_lilo",
     "headline"              :  UI(_("LILO")),
\end{verbatim}}%$

\subsection{Detection sequence}
At the key \verb+"detect_methods"+ the list of detection methods has to
be specified. The execution-sequence of the methods is equal to the 
specified sequence.

{\footnotesize \begin{verbatim}
     "detect_methods"   :  [
          ``(OSRLiloCheckPackage()),
          ``(OSRLiloConfTest())
     ],
\end{verbatim}}

\subsection{Module control and data dependencies}
At the keys \verb+"requires"+ and \verb+"provides"+ the information for
controlling the sequence of module-execution is defined. The framework
computes the sequence of the detection modules depending on the required
and known data. A module can only be executed if all the data it requires
is already known. This known data must have been provided by other modules
that were executed previously.\\

The required data are specified by an entry in the list at the key
\verb+"requires"+. All provided values from the existing modules can be
received by executing \verb+yast2 osr_provides+ and looking for the
file-contents in \verb+/tmp/osr_global_provides+.

{\footnotesize \begin{verbatim}
     "requires"         :  [
          "disk_device_list",
          "root_partition_device",
          "kernel_path_list",
          "initrd_path_list"
     ],
\end{verbatim}}

Values provided by this module are stored in the module-map at the key
\verb+"provides"+.

{\footnotesize \begin{verbatim}
     "provides"         : $[
          "lilo_package_is_installed" : $[
               "type"   : "boolean",
               "value"  : false,
               "status" : "unknown"
          ],
          "lilo_package_is_verified"  : $[
               "type"   : "boolean",
               "value"  : false,
               "status" : "unknown"
          ],
          "lilo_conf_path"            : $[
               "type"   : "string",
               "value"  : "/etc/lilo.conf",
               "status" : "unknown"
          ]
     ],
];
\end{verbatim}}

\section{Error detection and repair}
The sequence of the detection methods of a module is defined 
at the key \verb+"detect_methods"+ in the module-map. Detection methods not
mentioned here are not executed by the rescue system.\\

If an error is detected the detection method itself controls the further
procedure and executes the nesseccary repair methods.\\

Several functions are provided by the framework to integrate the modules
into the rescue system. See the ycpdoc-output in HTML format.
